\documentclass[]{book}
\usepackage{todonotes}
\usepackage{subfiles}
\usepackage{amsmath}
\usepackage[switch]{lineno}
\usepackage [english]{babel}
\usepackage [autostyle, english = american]{csquotes}
\MakeOuterQuote{"}

\newcommand{\foreign}[1]{\textit{#1}}
\newcommand{\undefine}[1]{\let#1\undefined}


\def\signed #1{{\leavevmode\unskip\nobreak\hfil\penalty50\hskip2em
  \hbox{}\nobreak\hfil(#1)%
  \parfillskip=0pt \finalhyphendemerits=0 \endgraf}}


\newcommand{\TOWRITE}{TODO: WRITE ME}
\newsavebox\mybox
\newenvironment{aquote}[1]
  {\savebox\mybox{#1}\begin{quote}}
  {\signed{\usebox\mybox}\end{quote}

  \quad

  \quad}

\title{\textbf{\LARGE Millions of Gestures}\\
       {\Large{Finding Meaning in the 21st Century}}}
\author{Sandy Maguire}
\date{}

\begin{document}

\frontmatter
\maketitle

\tableofcontents

\linenumbers
\mainmatter

\subfile{introduction}



\part{The Essential Attitude}
\chapter{No Miracle People}

\begin{aquote}{Richard Feynman}
  You ask me if an ordinary person, by studying hard, would get to be able to
  imagine these things like I imagine it -- of course! I was an ordinary person
  who studied hard. There's no miracle people. It just happens they got
  interested in this thing, and they learned all this stuff. They're just
  people. There's no talent or special miracle ability to understand quantum
  mechanics or miracle ability to imagine electromagnetic fields that comes
  without practice and reading and learning and study. So if you say you take an
  ordinary person who's willing to devote a great deal of time and study and
  work and thinking and mathematics and so on, then he's become a scientist.
\end{aquote}


\chapter{One Year to Live}
\TOWRITE

\subfile{part1/perfect-futures}
\subfile{part1/flinching}


\chapter{You Don't Have to Do What Other People Are Doing}

\begin{aquote}{Tynan, Life Nomadic}
  People will always warn you of the dangers of veering off the beaten path, but
  in their warnings you can see the fear that you might prove their worst
  nightmare true: that living your life on your own terms is not totally
  impossible. It’s a possibility that most people would rather not confront. The
  stakes are too high.
\end{aquote}

\section{All else being equal, strive for flexibility}
\TOWRITE


\part{Taking Stock}

\chapter{Character, Identity and Values}
\TOWRITE

\section{Who am I?}
\section{What are my core philosophies?}
\section{What sets me apart from other people?}
\section{Key strengths and weaknesses}
\section{Where is my dissatisfaction coming from?}
\section{What do I want out of life?}

\subfile{part2/location}
\subfile{part2/finances}


\chapter{Contribution and Impact}
\TOWRITE

\chapter{Skill Development}
\TOWRITE

\chapter{Social Life}
\TOWRITE

\chapter{Productivity and Organization}
\TOWRITE


\part{Forging a Path}

\chapter{Hamming Problems}
\TOWRITE

\chapter{Identifying Yourself}
\TOWRITE

\chapter{Doing What You Love}
\TOWRITE

\chapter{Prestige and Money}
\TOWRITE

\subfile{part3/minimalism}

\chapter{Neighborhoods / Putting Down Roots}
\TOWRITE

\chapter{Really Hard Goals}
\TOWRITE


\part{Moving Forward}

\chapter{Conclusion}
\TOWRITE

\chapter{Further Reading}
\TOWRITE

\end{document}
