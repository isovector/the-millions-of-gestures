\documentclass[../the-millions-of-gestures.tex]{subfiles}
\begin{document}

\chapter{Finances}
\TOWRITE
\section{Career}
\subsection{High Value / Low Reward, vv -- where are you?}
\subsection{Was that intentional?}
\subsection{Is this a good place for you to be?}
\subsection{Do you want to be "working?"}

\section{Finances}
\TOWRITE
\subsection{How much do you spend? Where? Why?}
\subsection{The retirement perspective}


Something I knew intellectually, but never really appreciated was just how damn
expensive it is to live in America. Like, it's actually crazy. For example, what
I paid in rent on my modest, one-room apartment in Denver was a little more than
three months' of full time work at minimum wage in the Baltics.

Right now I live on about \$25 dollars a day, which includes food, a casual beer
here and there, and living in the most gorgeous apartment I've ever had. And the
president of the country is my neighbor -- no kidding. We haven't socialized
yet, but I plan to invite her to board game night if I run into her on the
street. She probably won't come, but what do I have to lose?

Anyway, \todo{I'm not eating ramen.} I'm not particularly trying to be frugal
here, though I do try to keep my spending below \$10 a day unless I have a
really good reason not to.

This is a powerful thing to internalize. \$25 starts to feel like \emph{a lot
more money} when you realize you can live pretty lavishly for a day on it. All
of a sudden, that artisanal sandwich you paid \$18 for last week feels like even
more of a rip-off, doesn't it?

I found myself starting to measure purchases not in terms of "dollars" but in
"days of retirement." It's a lens that really puts life into perspective. A \$70
gym membership is obviously money well-spent, while \$600 to lease a car starts
to beg whether maybe car-ownership is really worth a month of your retirement.
Would I rather buy this \$7 beer or take the morning off?

\todo{focus on this point} It's pretty amazing what dumb shit we end up spending
money on, because it doesn't feel like a lot of dollars. But when viewed through
this lens, every dollar you spend takes an hour out of your retirement fund.
Damn. Let that sink in for a minute or two.

If you don't want to do any multiplication, I put together a (very) rough chart
corresponding to how much retirement you're giving up for some amount of money.
These numbers are more to give you an approximation and are designed to be
easy-to-remember rather than "correct."

\newcommand{\xx}[2]{\$#1 &\quad\iff\quad \text{#2}\\}
\begin{align*}
  \xx{1}{one hour}
  \xx{10}{one work day (8 hours)}
  \xx{25}{one full day (24 hours)}
  \xx{200}{one week}
  \xx{750}{one month}
  \xx{2\,000}{one quarter}
  \xx{4\,500}{half a year}
  \xx{10\,000}{a year}
\end{align*}
\undefine\xx

A good way to think about purchasing something is to first consult this chart,
and then seriously consider whether you'd rather have the thing in front of
your, or the corresponding amount of time to spend in retirement.

\subsection{GET OUT OF DEBT}
\subsection{If you were fired today, what is your game plan?}

\end{document}
