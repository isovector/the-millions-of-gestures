\documentclass[./the-millions-of-gestures.tex]{subfiles}
\begin{document}

\chapter*{Overture}
\addcontentsline{toc}{chapter}{Oveture}


\begin{aquote}{Chris Kelvin, Solaris, 2002}
  I followed the current. I was silent, attentive. I made a conscious effort to
  smile, nod, stand, and perform the millions of gestures that constitute life
  on earth. I studied these gestures until they became reflexes again.
\end{aquote}

\section*{Introduction}

The wisdom of our elders has always been an idea fascinating to me.

In my early twenties, I used to play a game. Whenever I had some time alone with
someone older than me, I'd ask them the question: "what's something you wished
you knew when you were my age?" The goal was one day to write a book about the
answers I got, but it became increasingly evident that nobody would read such a
book. The reason was simple: the answers I would get were always one of two
things.

"I wish I had been more confident."

"I wish I had known to start investing."

Good advice to be sure, but not exactly book-worthy. Everybody already knows
they should be investing (whether or not they do is a different beast
altogether), and telling someone to be more confident isn't very actionable. If
someone isn't already confident, telling them to be more so doesn't actually
help. If anything, it might hurt.

The concept of a quarter-life crisis is a relatively new one, identified as such
possibly beginning as late as Generation X. Since then, it seems only to be
growing in prevalence, particularly among young North American professionals.
According to the first Google result I came across, something like 75\% of
Millennial respondents on a Linkedin survey (aka "yuppies") said they had
experienced or were experiencing a quarter-life crisis.

Although I didn't respond to the Linkedin survey, I found myself (and the
majority of my friends) struggling through such a crisis for several years. I
was running away from my dissatisfaction, jumping from job to job, from city to
city, trying my best to randomly select the proper combination of variables I
thought would bring the much-desired meaning to my life. It goes without saying
that haphazardly flipping switches did nothing to assuage my problems.

In desperation, I took a month off of work and went traveling. It's cliche, I
know, but there might be a good reason why people look to traveling in an
attempt to find themselves. What I found, however, was just a bunch of people as
lost as I was. And we talked, sometimes for hours, helping one another over the
obstacles we'd already worked out for ourselves.

The idea of a sudden epiphany, I think, is mostly a myth. Big ideas don't come
from nowhere, they're constructed and refined, slowly and surely. Their seeds
come from a chance utterance of an acquaintance, or from an otherwise
unremarkable passage in a book. Big ideas are refined by going down to the cafe,
spending a lot of time thinking, writing in your notebook, and buying more
coffees than is likely healthy.  In other words, you need to be open to new
ideas, and when you see the glimmering of something great, you need to be
prepared to dive into the workshop of your mind and hammer it into something
worthwhile.

This was the insight I was missing. Instead of randomly changing variables in my
life hoping for something more, I should have been spending my time with a
notebook and a cup of strong coffee. I should have been approaching my life with
the same diligence I apply to my work.

\emph{We can be the authors of our own life stories.} It sounds obvious in
retrospect, but without a doubt, this is the most influential idea I've ever
come across. Agreeing the idea is obvious isn't enough though; it's
internalizing it that really counts. That's the hard part.

Without realizing it, all those years ago when I was asking people what they
wished they had known, I was looking for a particular answer. I didn't know what
it was, but there was something missing in my soul, something I was desperately
trying to find. It's this. We can be the authors of our own life stories.

We can, but it's hard. The system isn't exactly set up to help us do that.
Western culture, for all its good points, doesn't empower its adherents in this
way. \todo{empower twice} The American Dream is enticing -- if you work hard
enough, you can accumulate great wealth -- but this presupposes that great
wealth is a worthy thing to accumulate, and leads us to believe that it's
virtuous to work hard.

The American Dream is ubiquitous in North American culture, to the extent that
it's often invisible. While it's not a bad thing per se, we should
always be hesitant about reasoning in the presence of unseen influences. As a
cultural factor, it's clearly been successful, but we should keeping in mind
that it's very obviously a product of the Industrial Revolution.

Recall that this was a time characterized by working insanely long hours, under
terrible conditions, in order to stamp out indistinguishable goods, all in
service of becoming richer than the total sum of humans previously. Basing our
society, and by extension a large part of our identities, around the idea that
people are fungible, that the economy should be grown for its own sake, and that
hard work is desirable, seems, well, unwise to put it gently.

This strikes me as the primary cause of quarter-life and mid-life crises. I know
too many people who work jobs they hate, in order to make money they don't need,
so they can buy things they don't actually want. To a large degree, our society
defines success by how big our house is, what car we drive, how recently our
last promotion was, and by how successful our children are by the same metrics.

Many of us, myself included, pursue these metrics because these are the metrics
we are told to pursue. We buy a bigger house not because we really want one, but
because we feel like they could afford the mortgage payments on one. And,
probably, because we've been told that real-estate is always a good investment,
if not \emph{the best} investment.

So what can we do about all of this gloomy business? Far be it from me to only
acknowledge the problem without having a suggestion or two to offer. It comes,
however, with a caveat, which is that there is no one-size-fits-all answer to be
found here. What I \emph{don't} want to do is tell you "here's how to live
exactly the way I want to live." That's no good to you. Instead, we should focus
our efforts at a higher level, and look at strategies that will let you figure
out how to live exactly the way \emph{you} want.

In a very real way, this book is the answer I would give if someone asked me
what's something I wish I had known at that age. It's all of the things I wanted
to have been told.

\section*{Who Am I?}

In January of 2018, at the age of 27, I made the unconventional decision to
declare myself "retired indefinitely." This was somewhat of a shock to most
people in my life.

On paper, my life was going about as well as could be expected. I had a
prestigious degree from a prestigious university. I had worked at several of the
top companies in the world, and had strong recommendations from each of them. I
was well on my way to becoming a distinguished member of my field. By all
accounts, I had a bright career ahead of me.

But there was just one problem; that wasn't the life I wanted. What I wanted, I
wasn't sure, but it certainly wasn't that. The evidence of it was all around me;
with coworkers twenty years my senior working the same jobs as me.

I could all too easily imagine myself waking up at 50, wondering where the hell
my life had gone. I'd consult my bucket-list, and realize that I hadn't made any
progress on it in years, and now I had too many responsibilities and that my
health was too too poor to do the things I'd always promised myself I'd get
around to.

No thanks.

Despite my best intentions, I found myself having been sucked into the system. I
was working for the man in a job I hated. I was doing it to make money that I
didn't need, in order to buy things I didn't actually want. Like many others, I
had convinced myself that "this is just what being an adult is." I had resigned
myself to this unsatisfying life that I had unintentionally crafted for myself.

It wasn't the life I'd have designed for myself if I had been trying. And that's
because I hadn't been. If I was being honest with myself, I wasn't entirely sure
what that life would look like, but I was willing to find out. One of the
biggest problems of modern society is that it doesn't give us many chances to
try different things. Sure, you can order the salmon instead of the chicken when
you go out to dinner, but it's not like you can just decide to see what happens
if we give up democracy for a weekend or something. The existing order is too
deep-rooted and powerful for that.

But I wasn't happy where I was, and I didn't know where I wanted to be. The only
solution in that case is to experiment, and so I considered my retirement to be
an exercise in the scientific method.

The way I looked at it, it wasn't a particularly costly test. Worst case, it
wouldn't work out, and I could return to the society I left behind with my tail
between my legs, begging to be let back in. I had constructed that life once, by
accident nevertheless, so presumably I could rebuild it if I were actually
\emph{trying}.

Like many people, I have a bad habit of not being particularly good at putting
my money where my mouth is. \todo{terrible sentence} I'm sure that you hear
complaints weekly from your friends about their bad jobs, relationships or
what-have-you, but despite their comments that they're going to make a change,
things rarely do. Out of fear of continuing to live by status-quo forever, I
decided to pull the trigger, and quit my job immediately. I didn't have a plan;
I just knew I needed to make a change.

Three months later, I found myself living in Eastern Europe, spending 100\% of
my time working on meaningful personal projects, hanging out with people I love,
and accumulating a wealth of experiences I couldn't have even imagined existing
before moving here. It's been the best thing I've ever done with my life.

Maybe none of my story has resonated with you. If not, that's OK. This book
probably isn't for you, and I won't be offended if you put it down right here
and now. But maybe something has hit a chord in your spirit. Maybe you too have
some unnamed dissatisfaction in the direction your life has been taking.

Fortune favors the bold. If you're unhappy with some facet of your existence,
\emph{do something about it.} Don't let it fester within you, exploding in
twenty years in the form of a mid-life crisis. At that point it might be too
late. So take a chance, get outside of your comfort zone, and let's go for one
hell of a trip.

\end{document}

