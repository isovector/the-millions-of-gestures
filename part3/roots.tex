\documentclass[../the-millions-of-gestures.tex]{subfiles}
\begin{document}

\chapter{Putting Down Roots}

\begin{aquote}{Shirley Maclaine}
  The more I traveled the more I realized that fear makes strangers of people
  who should be friends.
\end{aquote}

Regardless of where you end up, be it where you started or a completely new
country, you're going to want to start settling in. If you've been living
somewhere for more than two months, and don't yet feel like you have a sense of
community, you're probably doing it wrong. Don't feel bad; as a generation,
collectively we're not particularly good at it.

At first glance, "putting down roots" doesn't seem like the kind of advice that
would be found in this book; it doesn't seem like advice that optimizes for
flexibility. But recall, if you've come this far, you've decided that, at least
on paper, this is the place you want to be. Whether or not it will pan out is a
different question, but it seems prudent to give this place the old college try,
doesn't it?

What really makes a place, however, is the people. Human beings, like it or not,
are social -- and yes, that even includes the antisocial ones among us. Barring
financial difficulties, if you're not in love with the place you live, it's a
good bet that you aren't well-enough integrated into the community.

So here's the plan. Starting today, you're going to meet two new people a day.
You'll be amazed at how quickly this adds up; there are what -- maybe a thousand
people who live in your neighborhood? That means that after a year, you'll know
a good majority of them.

It's hard to overstate how much of an effect this can have on your life. There's
a profound psychological experience that takes place when you know the people
around you. It can change the "place you live" into your "home". When you go
down to the coffee shop, you're bound to run into a few people that you know.
Recall that proximity and repeated exposure is a good recipe for making
friendships. Not necessarily strong ones, but if you've ever lived in a
dormitory, you know how much fun it can be to live in close proximity to a lot
of friends.

Two people a day. It probably sounds scary as hell, but it gets significantly
easier with each person. By day five, if you haven't been flinching out of it,
it will actually start feeling pretty good. By the end of the month, every time
you leave the house you'll reliably run into people you know.

It does a lot to relieve the inescapable feeling of isolation.

A good way to start meeting two people a day is to start slow and build your
success circles. Start by talking to the people you already reliably run into --
the baristas and locals at your favorite coffee shop or bar, for example. You
don't need to do anything big, just say hi and ask how their day is going. If
you get a positive reception, introduce yourself, say you live in the
neighborhood, and ask for their name. Success! There's one person for the day!

Caveat: make sure you remember their name. Use it the next time you see them,
just to prove to them you know what it is.

As you get better at this, challenge yourself to up the ante. Redefine what it
means to "meet somebody." Instead of just getting someone's name, challenge
yourself to maintain a five-minute conversation with someone before it counts.
Again, remember the details, and ask follow-up questions the next time you see
them. Not only will this positively influence your subjective experience, but
you can be sure as hell that people will remember you -- if for no reason other
than nobody else does this kind of thing.

Doing things that nobody else does is a powerful technique for standing out, and
standing out is a good way to make a strong impression.

\end{document}
