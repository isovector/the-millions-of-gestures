\documentclass[../the-millions-of-gestures.tex]{subfiles}
\begin{document}

\chapter{Stop Flinching}

\begin{aquote}{Julian Smith, The Flinch}
  The lessons you learn best are those you get burned by.  Without the scar,
  there’s no evidence or strong memory. The event didn’t actually happen or
  imprint itself on your brain—you just trusted those who know better. Adults
  know what’s safe, so you listen. Over a lifetime, those who listen too much
  build a habit of trust and conformity. Unfortunately, as time goes on, that
  habit becomes unbreakable.
\end{aquote}

\todo{needs lots of love} I'm looking out the window from my third floor
location, hidden somewhere in the depths of the university library. A few years
back I accidentally stumbled in here, and ever since it's been the place I come
to sit and think.

The students down there are running around, presumably headed off to their next
class. This university is known for its philosophy program. Without anything
else to go on, I'd guess the majority of them would appreciate what I'm doing
here. Thinking.

There's a kind of clockwork to life here. Every hour and a half, the otherwise
silent courtyard below me explodes with activity. Groups of people form for the
few minutes they can steal between classes, before disbanding. It's somehow
structured and chaotic simultaneously. It's beautiful.

To a large degree, the people here understand one another. They worry about the
same assignments, about getting home to see their friends and family for the
weekend, about whether or not their finances will allow them to party tonight
and still have enough to by the deluxe instant ramen next week, about what kinds
of work they'll find after university with their shiny new philosophy degrees.
Their lives run with the same cadences of weekly assignments and quarterly
finals.

Not only do they exist in the same context now, but they'll probably stay in
them. When I graduated from university, more than half of my cohort all moved to
the same city. Many of us found jobs together. We didn't feel the need to
integrate into our new home, because we had brought our old one with us. Whether
better or for worse, it's the shared contexts like these which build
communities.

And for many people, these organic communities are enough. They're familiar,
comfortable, safe. In many respects, they're boring, and I think that's a big
part of the appeal. That being said, despite all pretenses, the boring, safe
route isn't the only way to live. It isn't even necessarily the best way,
either.

For as long as I can remember, a powerful driving factor in my personal
development has been the motto of \emph{the best things in life are outside of
your comfort zone.} We're not ready to accept the most fantastic things life has
to offer. Statistically speaking, this has to be true. The space of human
experience is so large compared with the tiny number of things we have time to
do in our life.  Almost by necessity, we're not going to discover the best
things that existence has to offer us.

It's a litany by which I'm now trying to live. It's hard, though. Really hard.

Maybe you're a little like me. Maybe you too have countless stories where your
brain talked you out of doing something you wanted to. Maybe it was "I can't go
talk to that pretty girl standing by herself looking bored; I don't want to
inconvenience her," or perhaps something along the lines of "I shouldn't apply
for this job, because I know I won't get it." Our brains are duplicitous things,
and without constant checks and balances, they will usually lead us astray.

It's my desire to find the best things in life that has lead me here, to this
chair in the philosophy department on the third floor of this library. It's
quiet here. It's a good place to think. The students below have left the
courtyard, back to their classes. I'll be the first person to admit that I
haven't yet found the best things in life, nor am I anywhere close, nor will I
likely ever get there. But at least I'm looking.

I think that's as good a start as any.

We're all, I think, are held back by nothing but our own heads. "I can't do X.
I'm not good/smart/talented enough. I'm not the kind of person who does X." It
might be reassuring, or possibly terrifying, to realize that most of the people
out there who try things don't feel like they're good enough either. Usually
the only difference is that people out there do things are the people who try.
It's impostor syndrome at its finest.

Am I the right person to be writing a book on lifestyle design? Maybe. Maybe
not. Who can say, really? I'm just some guy who tried something different, and
who turned out to like it a hell of a lot more than the life he was being sold
by the standard institutions of Western culture. But in my opinion, it's a
worthwhile exercise, if only on the basis that it's outside of my comfort zone,
and therefore possibly one of the best things in life.

You never know until you try.

This is a book about trying.

\end{document}
