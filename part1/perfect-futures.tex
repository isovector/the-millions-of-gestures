\documentclass[../the-millions-of-gestures.tex]{subfiles}
\begin{document}

\chapter{Your Perfect Futures}

A few years back, I attended some kind of hokey meditation retreat. I didn't
remember signing up, but it was work-sanctioned so I decided to take it; it
seemed like a better way to spend a few days than slinging computer programs
around.

Throughout the retreat, I remember there being a lot of focus on breathing and
paying attention to what we were eating, and silly things like that. To be
honest, not much of it stuck with me. There was one exception.

The exercise that I remember was a meditation to visualize one of our perfect
futures. The instructor was clear to emphasize that we were looking for
\emph{one} of our perfect futures, implying there might be many, that it
might not be unique. This was already somewhat of a head-trip to me; I'd never
realized that there might be many, exceptionally different futures for me, all
of which I might describe as being perfect.

This was an epiphany, and being phrased like this allowed me to be open to
futures I would have otherwise ignored as being "out of hand." The human brain
is one that likes to feel smart, powerful and in-control of its own destiny, and
so if I had to choose \emph{the} perfect future, it would have been in-line
with all of the life choices I had made up to that point.

But instead I was given the great gift of needing to look only for any of my
perfect futures. Without much forethought, I put my pen to the page and just
started writing the stream of consciousness that had laid dormant but was now
flowing rapidly. I wrote without stopping for about five minutes, and fell
silent. The contents of my mind that now lay upon that paper honestly surprised
me.

The future I had come up with seemed very alien. I had envisioned myself living
a simple life in a commune on top of a mountain somewhere. I spent my time
growing food, talking with my neighbors, and tinkering with trains when I could.
Strange thoughts to be certain, but upon rereading them I was sure that this was
in fact a future with which I would be satisfied.

Curious, isn't it? That I could so easily come up with such a drastically
different way of living, one in which I was certainly happier than I was now?
It's almost as if I knew all along what I wanted, but as a defense mechanism,
had kept below the level of consciousness.

And so, if you're really and truly serious about wanting to change your life,
I'm going to make the very strong suggestion that you spend ten minutes thinking
about one of your ideal futures, and writing it down. If you are anything like
me when I read books, you might be thinking to yourself that you can just skip
the exercise and continue reading on. Don't do this. The lesson from education
is very clear about this: it's the homework that causes learning, not the
lecture itself. If you skip this (or any exercise recommended in this book) you
are doing yourself a disservice. Do the exercise.

The exercise is this: take ten minutes to write down two or three of your ideal
futures. Don't think about it, just set a timer, start writing, and stop when
the bell rings. If you find yourself describing a life that sounds a lot like
the one you're currently living, add in the first twist that comes to mind and
see where it takes you. The twist might be "what if I lived in a different
country" or "what if I had followed that passion I had as a child" or what have
you. If you are describing your current life, you are not being honest with
yourself, and following through with it will only delude you into fortifying the
status-quo with which you are unhappy.

Ready? Set your alarm for ten minutes. Start now.

Done.

Did you come up with anything surprising? It might not have been
earth-shattering, but I'll bet that upon review, at least one thing on your list
wasn't something you'd have expected to be there.

If so, that's great. I want you to focus on that surprising thing. The point of
the exercise is to realize that even if you're happy with the way things are
going, it's not the only way that things \emph{could} go. Keep this in mind.

\end{document}

