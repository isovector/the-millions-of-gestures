\documentclass[../the-millions-of-gestures.tex]{subfiles}
\begin{document}

\chapter{Minimalism}

What does the word "minimalism" mean to you?

\todo{monks evoke the right image, hippies dont}
\todo{story about schlepping boxes}

I'm not arguing that living minimally is the best strategy for everyone, always,
forever, but I am suggesting that it's a fantastic \emph{default} lifestyle
choice. The way that most people accumulate stuff is by accident; they'll buy
something on a whim, or receive it as a gift. By default, we keep things. We
hoard them.

There are a few obvious downsides to owning lots of stuff. The first is that
there is an implicit tax you need to pay for owning things. Although you'll have
more things, finding any \emph{particular} thing becomes harder the more junk
you have to search through. Furthermore, you actually need to keep that stuff
somewhere, which at best is taking up space, and at worst will actively get in
way of your lifestyle.

The second obvious downside to owning stuff is that it becomes an obstacle in
the way of changes. It's a lot easier to move to a new apartment if you can
leave all the furniture behind and only have two suitcases to your name. It's a
lot easier to go on vacation if nobody needs to sit your house and keep your dog
alive. These are not insurmountable problems by any means, but the easier a
change is to make, the more likely you are to do it. Not every change is an
improvement, but every improvement is a change, and that's worth keeping in
mind.

Last, and by far the most damning, the possession of things will keep you rooted
to your old self. The guitar that you bought on a whim and promised yourself
you'd learn how to play, but somehow never actually got around to. What good is
keeping that around? If you were actually serious about learning it, you would
have done it already. All that keeping it around serves is to remind you that
back then your will wasn't as strong as you wanted it to be. Get rid of it.

\todo{how often do you use each thing you own?}

So, take a few minutes to think about this. You are paying these taxes on
\emph{every single item you own.} You paid to buy it. You are paying to store
it. You are paying to have a harder time finding the things you want to find.
You are paying to be inconvenienced by it being in the way. You are paying to
move it, and to ensure it stays safe. You're paying to be distracted every time
you see it. That's pretty expensive. Unless you use that dang thing a few time a
week you're probably paying too much to own it.

This is why I say that minimalism is a good strategy by default. Unless you have
already spent a lot of time, thoughtfully curating exactly what you own, you own
too much crap. This isn't your fault by any means, it's more a reflection that
almost everything is crap. But that doesn't mean you're doomed to live with it
forever. You're the author of your own life story. If you don't have any other
plan, aim for minimalism. Worst case, you'll realize it's not for you after a
few months, and you'll be a lot more mindful of the things you do acquire in the
future.

Related to this point, about how owning things anchors you to the person you
were when you bought them, is another way in which we can explore the concept of
"minimalism." That's the idea of keeping your identity small. Identities are
very powerful things, and they act as strong filters on what kind of things we
take seriously. Whenever we're exposed to something new, there's a good chance
we'll dismiss it out of hand. "I'm not the kind of person who does X."

Remember, all the best things in life are outside of our comfort zone. Without
careful gardening, our identities will serve only to keep us \emph{inside} of
our comfort zones.

\end{document}

