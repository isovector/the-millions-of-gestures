\documentclass[]{book}
\usepackage{todonotes}
\usepackage{amsmath}
\usepackage[switch]{lineno}
\usepackage [english]{babel}
\usepackage [autostyle, english = american]{csquotes}
\MakeOuterQuote{"}

\newcommand{\foreign}[1]{\textit{#1}}
\newcommand{\undefine}[1]{\let#1\undefined}


\def\signed #1{{\leavevmode\unskip\nobreak\hfil\penalty50\hskip2em
  \hbox{}\nobreak\hfil(#1)%
  \parfillskip=0pt \finalhyphendemerits=0 \endgraf}}

\newsavebox\mybox
\newenvironment{aquote}[1]
  {\savebox\mybox{#1}\begin{quote}}
  {\signed{\usebox\mybox}\end{quote}}

\title{\textbf{\LARGE Millions of Gestures}\\
       {\Large{Finding Meaning in the 21st Century}}}
\author{Sandy Maguire}
\date{}

\begin{document}

\frontmatter
\maketitle

\tableofcontents

\linenumbers

\chapter*{Preface}
\addcontentsline{toc}{chapter}{Preface}

The wisdom of our elders has always been an idea fascinating to me.

In my early twenties, I used to play a game. Whenever I had some time alone with
someone older than me, I'd ask them the question: "what's something you wished
you knew when you were my age?" The goal was one day to write a book about the
answers I got, but it became increasingly evident that nobody would read such a
book. The reason was simple: the answers I would get were always one of two
things.

"I wish I had been more confident."

"I wish I had known to start investing."

Good advice to be sure, but not exactly book-worthy. Everybody already knows
they should be investing (whether or not they do is a different beast
altogether), and telling someone to be more confident isn't very actionable. If
someone isn't already confident, telling them to be more so doesn't actually
help. If anything, it might hurt.

The concept of a quarter-life crisis is a relatively new one, identified as such
possibly beginning as late as Generation X. Since then, it seems only to be
growing in prevalence, particularly among young North American professionals.
According to the first Google result I came across, something like 75\% of
Millennial respondents on a Linkedin survey (aka "yuppies") said they had
experienced or were experiencing a quarter-life crisis.

Although I didn't respond to the Linkedin survey, I found myself (and the
majority of my friends) struggling through such a crisis for several years. I
was running away from my dissatisfaction, jumping from job to job, from city to
city, trying my best to randomly select the proper combination of variables I
thought would bring the much-desired meaning to my life. It goes without saying
that haphazardly flipping switches did nothing to assuage my problems.

In desperation, I took a month off of work and went traveling. It's cliche, I
know, but there might be a good reason why people look to traveling in an
attempt to find themselves. What I found, however, was just a bunch of people as
lost as I was. And we talked, sometimes for hours, helping one another over the
obstacles we'd already worked out for ourselves.

The idea of a sudden epiphany, I think, is mostly a myth. Big ideas don't come
from nowhere, they're constructed and refined, slowly and surely. Their seeds
come from a chance utterance of an acquaintance, or from an otherwise
unremarkable passage in a book. Big ideas are refined by going down to the cafe,
spending a lot of time thinking, writing in your notebook, and buying more
coffees than is likely healthy.  In other words, you need to be open to new
ideas, and when you see the glimmering of something great, you need to be
prepared to dive into the workshop of your mind and hammer it into something
worthwhile.

This was the insight I was missing. Instead of randomly changing variables in my
life hoping for something more, I should have been spending my time with a
notebook and a cup of strong coffee. I should have been approaching my life with
the same diligence I apply to my work.

\emph{We can be the authors of our own life stories.} It sounds obvious in
retrospect, but without a doubt, this is the most influential idea I've ever
come across. Agreeing the idea is obvious isn't enough though; it's
internalizing it that really counts. That's the hard part.

Without realizing it, all those years ago when I was asking people what they
wished they had known, I was looking for a particular answer. I didn't know what
it was, but there was something missing in my soul, something I was desperately
trying to find. It's this. We can be the authors of our own life stories.

We can, but it's hard. The system isn't exactly set up to help us do that.
Western culture, for all its good points, doesn't empower its adherents in this
way. \todo{empower twice} The American Dream is enticing -- if you work hard
enough, you can accumulate great wealth -- but this presupposes that great
wealth is a worthy thing to accumulate, and leads us to believe that it's
virtuous to work hard.

The American Dream is ubiquitous in North American culture, to the extent that
it's often invisible. While it's not a bad thing per se, we should
always be hesitant about reasoning in the presence of unseen influences. As a
cultural factor, it's clearly been successful, but we should keeping in mind
that it's very obviously a product of the Industrial Revolution.

Recall that this was a time characterized by working insanely long hours, under
terrible conditions, in order to stamp out indistinguishable goods, all in
service of becoming richer than the total sum of humans previously. Basing our
society, and by extension a large part of our identities, around the idea that
people are fungible, that the economy should be grown for its own sake, and that
hard work is desirable, seems, well, unwise to put it gently.

This strikes me as the primary cause of quarter-life and mid-life crises. I know
too many people who work jobs they hate, in order to make money they don't need,
so they can buy things they don't actually want. To a large degree, our society
defines success by how big our house is, what car we drive, how recently our
last promotion was, and by how successful our children are by the same metrics.

Many of us, myself included, pursue these metrics because these are the metrics
we are told to pursue. We buy a bigger house not because we really want one, but
because we feel like they could afford the mortgage payments on one. And,
probably, because we've been told that real-estate is always a good investment,
if not \emph{the best} investment.

So what can we do about all of this gloomy business? Far be it from me to only
acknowledge the problem without having a suggestion or two to offer. It comes,
however, with a caveat, which is that there is no one-size-fits-all answer to be
found here. What I \emph{don't} want to do is tell you "here's how to live
exactly the way I want to live." That's no good to you. Instead, we should focus
our efforts at a higher level, and look at strategies that will let you figure
out how to live exactly the way \emph{you} want.

In a very real way, this book is the answer I would give if someone asked me
what's something I wish I had known at that age. It's all of the things I wanted
to have been told.


\mainmatter


\chapter{Overture}

\begin{aquote}{Chris Kelvin, Solaris, 2002}
  I followed the current. I was silent, attentive. I made a conscious effort to
  smile, nod, stand, and perform the millions of gestures that constitute life
  on earth. I studied these gestures until they became reflexes again.
\end{aquote}

\qquad

\qquad

I'm looking out the window from my third floor location, hidden somewhere in the
depths of the university library. A few years back I accidentally stumbled in
here, and ever since it's been the place I come to sit and think.

The students down there are running around, presumably headed off to their next
class. This university is known for its philosophy program. Without anything
else to go on, I'd guess the majority of them would appreciate what I'm doing
here. Thinking.

There's a kind of clockwork to life here. Every hour and a half, the otherwise
silent courtyard below me explodes with activity. Groups of people form for the
few minutes they can steal between classes, before disbanding. It's somehow
structured and chaotic simultaneously. It's beautiful.

To a large degree, the people here understand one another. They worry about the
same assignments, about getting home to see their friends and family for the
weekend, about whether or not their finances will allow them to party tonight
and still have enough to by the deluxe instant ramen next week, about what kinds
of work they'll find after university with their shiny new philosophy degrees.
Their lives run with the same cadences of weekly assignments and quarterly
finals.

Not only do they exist in the same context now, but they'll probably stay in
them. When I graduated from university, more than half of my cohort all moved to
the same city. Many of us found jobs together. We didn't feel the need to
integrate into our new home, because we had brought our old one with us. Whether
better or for worse, it's the shared contexts like these which build
communities.

And for many people, these organic communities are enough. They're familiar,
comfortable, safe. In many respects, they're boring, and I think that's a big
part of the appeal. That being said, despite all pretenses, the boring, safe
route isn't the only way to live. It isn't even necessarily the best way,
either.

For as long as I can remember, a powerful driving factor in my personal
development has been the motto of \emph{the best things in life are outside of
your comfort zone.} We're not ready to accept the most fantastic things life has
to offer. Statistically speaking, this has to be true. The space of human
experience is so large compared with the tiny number of things we have time to
do in our life.  Almost by necessity, we're not going to discover the best
things that existence has to offer us.

It's a litany by which I'm now trying to live. It's hard, though. Really hard.

Maybe you're a little like me. Maybe you too have countless stories where your
brain talked you out of doing something you wanted to. Maybe it was "I can't go
talk to that pretty girl standing by herself looking bored; I don't want to
inconvenience her," or perhaps something along the lines of "I shouldn't apply
for this job, because I know I won't get it." Our brains are duplicitous things,
and without constant checks and balances, they will usually lead us astray.

It's my desire to find the best things in life that has lead me here, to this
chair in the philosophy department on the third floor of this library. It's
quiet here. It's a good place to think. The students below have left the
courtyard, back to their classes. I'll be the first person to admit that I
haven't yet found the best things in life, nor am I anywhere close, nor will I
likely ever get there. But at least I'm looking.

I think that's as good a start as any.

We're all, I think, are held back by nothing but our own heads. "I can't do X.
I'm not good/smart/talented enough. I'm not the kind of person who does X." It
might be reassuring, or possibly terrifying, to realize that most of the people
out there who try things don't feel like they're good enough either. Usually
the only difference is that people out there do things are the people who try.
It's impostor syndrome at its finest.

Am I the right person to be writing a book on lifestyle design? Maybe. Maybe
not. Who can say, really? I'm just some guy who tried something different, and
who turned out to like it a hell of a lot more than the life he was being sold
by the standard institutions of Western culture. But in my opinion, it's a
worthwhile exercise, if only on the basis that it's outside of my comfort zone,
and therefore possibly one of the best things in life.

You never know until you try.

This is a book about trying.


\chapter{Starting Again}

In January of 2018, at the age of 27, I made the unconventional decision to
declare myself "retired indefinitely." This was somewhat of a shock to most
people in my life.

On paper, my life was going about as well as could be expected. I had a
prestigious degree from a prestigious university. I had worked at several of the
top companies in the world, and had strong recommendations from each of them. I
was well on my way to becoming a distinguished member of my field. By all
accounts, I had a bright career ahead of me.

But there was just one problem; that wasn't the life I wanted. What I wanted, I
wasn't sure, but it certainly wasn't that. The evidence of it was all around me;
with coworkers twenty years my senior working the same jobs as me.

I could all too easily imagine myself waking up at 50, wondering where the hell
my life had gone. I'd consult my bucket-list, and realize that I hadn't made any
progress on it in years, and now I had too many responsibilities and that my
health was too too poor to do the things I'd always promised myself I'd get
around to.

No thanks.

Despite my best intentions, I found myself having been sucked into the system. I
was working for the man in a job I hated. I was doing it to make money that I
didn't need, in order to buy things I didn't actually want. Like many others, I
had convinced myself that "this is just what being an adult is." I had resigned
myself to this unsatisfying life that I had unintentionally crafted for myself.

It wasn't the life I'd have designed for myself if I had been trying. And that's
because I hadn't been. If I was being honest with myself, I wasn't entirely sure
what that life would look like, but I was willing to find out. One of the
biggest problems of modern society is that it doesn't give us many chances to
try different things. Sure, you can order the salmon instead of the chicken when
you go out to dinner, but it's not like you can just decide to see what happens
if we give up democracy for a weekend or something. The existing order is too
deep-rooted and powerful for that.

But I wasn't happy where I was, and I didn't know where I wanted to be. The only
solution in that case is to experiment, and so I considered my retirement to be
an exercise in the scientific method.

The way I looked at it, it wasn't a particularly costly test. Worst case, it
wouldn't work out, and I could return to the society I left behind with my tail
between my legs, begging to be let back in. I had constructed that life once, by
accident nevertheless, so presumably I could rebuild it if I were actually
\emph{trying}.

Like many people, I have a bad habit of not being particularly good at putting
my money where my mouth is. \todo{terrible sentence} I'm sure that you hear
complaints weekly from your friends about their bad jobs, relationships or
what-have-you, but despite their comments that they're going to make a change,
things rarely do. Out of fear of continuing to live by status-quo forever, I
decided to pull the trigger, and quit my job immediately. I didn't have a plan;
I just knew I needed to make a change.

Three months later, I found myself living in Eastern Europe, spending 100\% of
my time working on meaningful personal projects, hanging out with people I love,
and accumulating a wealth of experiences I couldn't have even imagined existing
before moving here. It's been the best thing I've ever done with my life.

Maybe none of my story has resonated with you. If not, that's OK. This book
probably isn't for you, and I won't be offended if you put it down right here
and now. But maybe something has hit a chord in your spirit. Maybe you too have
some unnamed dissatisfaction in the direction your life has been taking.

Fortune favors the bold. If you're unhappy with some facet of your existence,
\emph{do something about it.} Don't let it fester within you, exploding in
twenty years in the form of a mid-life crisis. At that point it might be too
late. So take a chance, get outside of your comfort zone, and let's go for one
hell of a trip.


\chapter{Taking Stock}

A few years back, I attended some kind of hokey meditation retreat. I didn't
remember signing up, but it was work-sanctioned so I decided to take it; it
seemed like a better way to spend a few days than slinging computer programs
around.

Throughout the retreat, I remember there being a lot of focus on breathing and
paying attention to what we were eating, and silly things like that. To be
honest, not much of it stuck with me. There was one exception.

The exercise that I remember was a meditation to visualize one of our perfect
futures. The instructor was clear to emphasize that we were looking for
\emph{one} of our perfect futures, implying there might be many, that it
might not be unique. This was already somewhat of a head-trip to me; I'd never
realized that there might be many, exceptionally different futures for me, all
of which I might describe as being perfect.

This was an epiphany, and being phrased like this allowed me to be open to
futures I would have otherwise ignored as being "out of hand." The human brain
is one that likes to feel smart, powerful and in-control of its own destiny, and
so if I had to choose \emph{the} perfect future, it would have been in-line
with all of the life choices I had made up to that point.

But instead I was given the great gift of needing to look only for any of my
perfect futures. Without much forethought, I put my pen to the page and just
started writing the stream of consciousness that had laid dormant but was now
flowing rapidly. I wrote without stopping for about five minutes, and fell
silent. The contents of my mind that now lay upon that paper honestly surprised
me.

The future I had come up with seemed very alien. I had envisioned myself living
a simple life in a commune on top of a mountain somewhere. I spent my time
growing food, talking with my neighbors, and tinkering with trains when I could.
Strange thoughts to be certain, but upon rereading them I was sure that this was
in fact a future with which I would be satisfied.

Curious, isn't it? That I could so easily come up with such a drastically
different way of living, one in which I was certainly happier than I was now?
It's almost as if I knew all along what I wanted, but as a defense mechanism,
had kept below the level of consciousness.

And so, if you're really and truly serious about wanting to change your life,
I'm going to make the very strong suggestion that you spend ten minutes thinking
about one of your ideal futures, and writing it down. If you are anything like
me when I read books, you might be thinking to yourself that you can just skip
the exercise and continue reading on. Don't do this. The lesson from education
is very clear about this: it's the homework that causes learning, not the
lecture itself. If you skip this (or any exercise recommended in this book) you
are doing yourself a disservice. Do the exercise.

The exercise is this: take ten minutes to write down two or three of your ideal
futures. Don't think about it, just set a timer, start writing, and stop when
the bell rings. If you find yourself describing a life that sounds a lot like
the one you're currently living, add in the first twist that comes to mind and
see where it takes you. The twist might be "what if I lived in a different
country" or "what if I had followed that passion I had as a child" or what have
you. If you are describing your current life, you are not being honest with
yourself, and following through with it will only delude you into fortifying the
status-quo with which you are unhappy.

Ready? Set your alarm for ten minutes. Start now.

Done.

Did you come up with anything surprising? It might not have been
earth-shattering, but I'll bet that upon review, at least one thing on your list
wasn't something you'd have expected to be there.

If so, that's great. I want you to focus on that surprising thing. The point of
the exercise is to realize that even if you're happy with the way things are
going, it's not the only way that things \emph{could} go. Keep this in mind.

\todo{think about what you're good at}
\todo{what are your comparative advantages?}
\todo{what do you want to dedicate your life to?}

\todo{maybe this should be more 8760 hours kind of stuff}
\todo{and then move this chapter + some goal factoring stuff into its own}



\chapter{Minimalism}

What does the word "minimalism" mean to you?

\todo{monks evoke the right image, hippies dont}
\todo{story about schlepping boxes}

I'm not arguing that living minimally is the best strategy for everyone, always,
forever, but I am suggesting that it's a fantastic \emph{default} lifestyle
choice. The way that most people accumulate stuff is by accident; they'll buy
something on a whim, or receive it as a gift. By default, we keep things. We
hoard them.

There are a few obvious downsides to owning lots of stuff. The first is that
there is an implicit tax you need to pay for owning things. Although you'll have
more things, finding any \emph{particular} thing becomes harder the more junk
you have to search through. Furthermore, you actually need to keep that stuff
somewhere, which at best is taking up space, and at worst will actively get in
way of your lifestyle.

The second obvious downside to owning stuff is that it becomes an obstacle in
the way of changes. It's a lot easier to move to a new apartment if you can
leave all the furniture behind and only have two suitcases to your name. It's a
lot easier to go on vacation if nobody needs to sit your house and keep your dog
alive. These are not insurmountable problems by any means, but the easier a
change is to make, the more likely you are to do it. Not every change is an
improvement, but every improvement is a change, and that's worth keeping in
mind.

Last, and by far the most damning, the possession of things will keep you rooted
to your old self. The guitar that you bought on a whim and promised yourself
you'd learn how to play, but somehow never actually got around to. What good is
keeping that around? If you were actually serious about learning it, you would
have done it already. All that keeping it around serves is to remind you that
back then your will wasn't as strong as you wanted it to be. Get rid of it.

\todo{how often do you use each thing you own?}

So, take a few minutes to think about this. You are paying these taxes on
\emph{every single item you own.} You paid to buy it. You are paying to store
it. You are paying to have a harder time finding the things you want to find.
You are paying to be inconvenienced by it being in the way. You are paying to
move it, and to ensure it stays safe. You're paying to be distracted every time
you see it. That's pretty expensive. Unless you use that dang thing a few time a
week you're probably paying too much to own it.

This is why I say that minimalism is a good strategy by default. Unless you have
already spent a lot of time, thoughtfully curating exactly what you own, you own
too much crap. This isn't your fault by any means, it's more a reflection that
almost everything is crap. But that doesn't mean you're doomed to live with it
forever. You're the author of your own life story. If you don't have any other
plan, aim for minimalism. Worst case, you'll realize it's not for you after a
few months, and you'll be a lot more mindful of the things you do acquire in the
future.

Related to this point, about how owning things anchors you to the person you
were when you bought them, is another way in which we can explore the concept of
"minimalism." That's the idea of keeping your identity small. Identities are
very powerful things, and they act as strong filters on what kind of things we
take seriously. Whenever we're exposed to something new, there's a good chance
we'll dismiss it out of hand. "I'm not the kind of person who does X."

Remember, all the best things in life are outside of our comfort zone. Without
careful gardening, our identities will serve only to keep us \emph{inside} of
our comfort zones.



\chapter{Finding a Space}

Let's begin with an analogy.

Why are you friends with the people you're friends with? At first blush it seems
like a silly question, but there's some meat behind it. If you're like most
people (myself included!), the people you've chosen to be friends with are
mostly accidental. Usually, they're people who were in close physical proximity
to you -- maybe coworkers, neighbors, friends of friends, etc -- people whom you
saw often enough to become familiar with. And, all else being equal, these were
people you didn't dislike enough to disassociate with.

As it happens, this is usually enough. If you spend enough time with someone you
don't hate, eventually you'll become friends. Maybe not "best friends", but
someone you'd feel obligated to talk to if you ran into on the street. Someone
you wouldn't mind going out for beers with once in a while. Someone you might
call up to do something if you were feeling bored.

Of course, there are notable exceptions. You have best friends and significant
others, and generally these are people that either you or they put "more effort"
into making happen. Maybe you met, decided you liked their style, and made a
move. From my own experience, my best relationships have been those where I've
identified an acquaintance whom I really like "on paper", and have decided to
tell them that I think they're super cool and I'd love to become better friends.
Often times it doesn't go anywhere, but when it does, it's truly something
marvelous.

\todo{proximity friends get the job done, but they're not satisfying. happiness
without joy}

There's something to be said about this. If you accept my premise, the
difference between best friends -- those people you'd go out of your way to
spend time with -- and your everyday acquaintance-friend is the degree to which
you and this other person decided to make it happen. For one reason or another,
you saw enough in the other that the investment into becoming great friends
seemed worthwhile.

I don't really want to talk about friends. I want to talk about finding a space
for yourself in this world. Not in a metaphorical sense, but like,
geographically.

The friends you made due to being in proximity with them are fine. They get the
job done. But, for the most part, you're friends by accident. In the same way,
the place you currently live is probably incidental as well. Maybe you were born
there, maybe you moved there for a job, or for a relationship. But most of the
time, for most of the people, they're not living somewhere because they sat down
and decided it would be the best place for them to live.

That's kind of insane, if you think about it.
\todo{I think maybe scrap all of above this.}

It can't be denied that places have their own character, their own personas. For
better or for worse,



\chapter{Financing Yourself}

\todo{what do we mean by retirement?}

Something I knew intellectually, but never really appreciated was just how damn
expensive it is to live in America. Like, it's actually crazy. For example, what
I paid in rent on my modest, one-room apartment in Denver was a little more than
three months' of full time work at minimum wage in the Baltics.

Right now I live on about \$25 dollars a day, which includes food, a casual beer
here and there, and living in the most gorgeous apartment I've ever had. And the
president of the country is my neighbor -- no kidding. We haven't socialized
yet, but I plan to invite her to board game night if I run into her on the
street. She probably won't come, but what do I have to lose?

Anyway, \todo{I'm not eating ramen.} I'm not particularly trying to be frugal
here, though I do try to keep my spending below \$10 a day unless I have a
really good reason not to.

This is a powerful thing to internalize. \$25 starts to feel like \emph{a lot
more money} when you realize you can live pretty lavishly for a day on it. All
of a sudden, that artisanal sandwich you paid \$18 for last week feels like even
more of a rip-off, doesn't it?

I found myself starting to measure purchases not in terms of "dollars" but in
"days of retirement." It's a lens that really puts life into perspective. A \$70
gym membership is obviously money well-spent, while \$600 to lease a car starts
to beg whether maybe car-ownership is really worth a month of your retirement.
Would I rather buy this \$7 beer or take the morning off?

\todo{focus on this point} It's pretty amazing what dumb shit we end up spending
money on, because it doesn't feel like a lot of dollars. But when viewed through
this lens, every dollar you spend takes an hour out of your retirement fund.
Damn. Let that sink in for a minute or two.

If you don't want to do any multiplication, I put together a (very) rough chart
corresponding to how much retirement you're giving up for some amount of money.
These numbers are more to give you an approximation and are designed to be
easy-to-remember rather than "correct."

\newcommand{\xx}[2]{\$#1 &\quad\iff\quad \text{#2}\\}
\begin{align*}
  \xx{1}{one hour}
  \xx{10}{one work day (8 hours)}
  \xx{25}{one full day (24 hours)}
  \xx{200}{one week}
  \xx{750}{one month}
  \xx{2\,000}{one quarter}
  \xx{4\,500}{half a year}
  \xx{10\,000}{a year}
\end{align*}
\undefine\xx

A good way to think about purchasing something is to first consult this chart,
and then seriously consider whether you'd rather have the thing in front of
your, or the corresponding amount of time to spend in retirement.



\chapter{Putting Down Roots}

rich american + cultural appropriation? no, go fuck yourself

if you make an effort to fit in, people will love you.

learn the local language, follow the local culture, make friends with locals

the goal is to integrate. it's ok to have expat friends, but keep them under 50\%


\end{document}

