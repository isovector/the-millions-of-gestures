\documentclass[]{book}
\usepackage{todonotes}
\usepackage [english]{babel}
\usepackage [autostyle, english = american]{csquotes}
\MakeOuterQuote{"}


\def\signed #1{{\leavevmode\unskip\nobreak\hfil\penalty50\hskip2em
  \hbox{}\nobreak\hfil(#1)%
  \parfillskip=0pt \finalhyphendemerits=0 \endgraf}}

\newsavebox\mybox
\newenvironment{aquote}[1]
  {\savebox\mybox{#1}\begin{quote}}
  {\signed{\usebox\mybox}\end{quote}}

%opening
\title{The Millions of Gestures}
\author{Sandy Maguire}

\begin{document}

\frontmatter
\maketitle

\tableofcontents

\mainmatter

\chapter{Introduction}

\begin{aquote}{Chris Kelvin, Solaris, 2002}
  I followed the current. I was silent, attentive. I made a conscious effort to
  smile, nod, stand, and perform the million of gestures that constitute life on
  earth. I studied these gestures until they became reflexes again.
\end{aquote}


\chapter{Starting Again}

In January of 2018, at the age of 27, I made the unconventional decision to
declare myself "retired indefinitely." This was somewhat of a shock to most
people in my life.

On paper, my life was going about as well as could be expected. I had a
prestigious degree from a prestigious university. I had worked at several of the
top companies in the world, and had strong recommendations from each of them. I
was well on my way to becoming a distinguished member of my field. By all
accounts, I had a bright career ahead of me.

But there was just one problem; that wasn't the life I wanted. What I wanted, I
wasn't sure, but it certainly wasn't that. The evidence of it was all around me;
with coworkers twenty years my senior working the same jobs as me.

I could too easily imagine myself waking up at 50, wondering where the hell my
life had gone. I'd consult my bucket-list, and realize that I hadn't made any
progress on it in years, and now I had too many responsibilities and too poor
health to do the things I'd always promised myself I'd do.

No thanks.

Despite my best intentions, I found myself having been sucked into the system. I
was working for the man in a job I hated. I was doing it to make money that I
didn't need, in order to buy things I didn't actually want. Like many others, I
had convinced myself that "this is just what being an adult is." I had resigned
myself to this unsatisfying life that I had unintentionally crafted for myself.

It wasn't the life I'd have designed for myself if I had been trying. If I was
being honest with myself, , I wasn't entirely sure what that life would look
like, but I was willing to find out. One of the biggest problems of modern
society is that it doesn't give us many chances to try different things. Sure,
you can order the salmon instead of the chicken when you go out to dinner, but
it's not like you can just decide to see what happens if we give up democracy
for a weekend or something.  The existing order is too deep-rooted and powerful
for that.

But I wasn't happy where I was, and I didn't know where I wanted to be. The only
solution in that case is to experiment, and so I considered my retirement to be
an exercise in the scientific method.

The way I looked at it, it wasn't a particularly costly test. Worst case, it
wouldn't work out, and I could return to the society I left behind with my tail
between my legs, begging to be let back in. I had constructed that life once, by
accident nevertheless, so presumably I could rebuild it if I were actually
\textit{trying}.

Like many people, I have a bad habit of not being particularly good at putting
my money where my mouth is. \todo{terrible sentence} I'm sure that you hear
complaints weekly from your friends about their bad jobs, relationships or
what-have-you, but despite their comments that they're going to make a change,
things rarely do. Out of fear of continuing to live by status-quo forever, I
decided to pull the trigger, and quit my job immediately. I didn't have a plan;
I just knew I needed to make a change.

Three months later, I found myself living in Eastern Europe, spending 100\% of
my time working on meaningful personal projects, hanging out with people I love,
and accumulating a wealth of experiences I couldn't have even imagined existing
before moving here. It's been the best thing I've ever done with my life.

Maybe none of my story has resonated with you. If not, that's OK. This book
probably isn't for you, and I won't be offended if you put it down right here
and now. But maybe something has hit a chord in your spirit. Maybe you too have
some unnamed dissatisfaction in the direction your life has been taking.

Fortune favors the bold. If you're unhappy with some facet of your existence,
\textit{do something about it.} Don't let it fester within you, exploding in
twenty years in the form of a mid-life crisis. At that point it might be too
late. So take a chance, get outside of your comfort zone, and let's go for one
hell of a trip.


\chapter{Taking Stock}

A few years back, I attended some kind of hokey meditation retreat. I didn't
remember signing up, but it was work-sanctioned so I decided to take it; it
seemed like a better way to spend a few days than slinging computer programs
around.

Throughout the retreat, I remember there being a lot of focus on breathing and
paying attention to what we were eating, and silly things like that. To be
honest, not much of it stuck with me. There was one exception.

The exercise that I remember was a meditation to visualize one of our perfect
futures. The instructor was clear to emphasize that we were looking for
\textit{one} of our perfect futures, implying there might be many, that it might
not be unique. This was already somewhat of a head-trip to me; I'd never
realized that there might be many, exceptionally different futures for me, all
of which I might describe as being perfect.

This was an epiphany to me, and being phrased like this allowed me to be open to
futures I would have otherwise ignored as being "out of hand." The human brain
is one that likes to feel smart, powerful and in-control of its own destiny, and
so if I had to choose \textit{the} perfect future, it would have been in-line
with all of the life choices I had made up to that point.

But instead I was given the great gift of needing to look only for any of my
perfect futures. Without much forethought, I put my pen to the page and just
started writing the stream of consciousness that had laid dormant but was now
flowing rapidly. I wrote without stopping for about five minutes, and fell
silent. The contents of my mind that now lay upon that paper honestly surprised
me.

The future I had come up with seemed very alien. I had envisioned myself living
a simple life in a commune on top of a mountain somewhere. I spent my time
growing food, talking with my neighbors, and tinkering with trains when I could.
Strange thoughts to be certain, but upon rereading them I was sure that this was
in fact a future with which I would be satisfied.

Curious, isn't it? That I could so easily come up with such a drastically
different way of living, one in which I was certainly happier than I was now?
It's almost as if I knew all along what I wanted, but as a defense mechanism,
had kept below the level of consciousness.

And so, if you're really and truly serious about wanting to change your life,
I'm going to make the very strong suggestion that you spend ten minutes thinking
about one of your ideal futures, and writing it down. If you are anything like
me when I read books, you might be thinking to yourself that you can just skip
the exercise and continue reading on. Don't do this. The lesson from education
is very clear about this: it's the homework that causes learning, not the
lecture itself. If you skip this (or any exercise recommended in this book) you
are doing yourself a disservice. Do the exercise.

The exercise is this: take ten minutes to write down two or three of your ideal
futures. Don't think about it, just set a timer, start writing, and stop when
the bell rings. If you find yourself describing a life that sounds a lot like
the one you're currently living, add in the first twist that comes to mind and
see where it takes you. The twist might be "what if I lived in a different
country" or "what if I had followed that passion I had as a child" or what have
you. If you are describing your current life, you are not being honest with
yourself, and following through with it will only delude you into fortifying the
status-quo with which you are unhappy.

Ready? Set your alarm for ten minutes. Start now.


\end{document}

