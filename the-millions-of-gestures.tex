\documentclass[]{book}
\usepackage{todonotes}
\usepackage[switch]{lineno}
\usepackage [english]{babel}
\usepackage [autostyle, english = american]{csquotes}
\MakeOuterQuote{"}

\newcommand{\foreign}[1]{\textit{#1}}
\newcommand{\emphasis}[1]{\textit{#1}}


\def\signed #1{{\leavevmode\unskip\nobreak\hfil\penalty50\hskip2em
  \hbox{}\nobreak\hfil(#1)%
  \parfillskip=0pt \finalhyphendemerits=0 \endgraf}}

\newsavebox\mybox
\newenvironment{aquote}[1]
  {\savebox\mybox{#1}\begin{quote}}
  {\signed{\usebox\mybox}\end{quote}}

\title{\textbf{\LARGE The Millions of Gestures}\\{\Large Navigating a
Quarter-life Crisis}}
\author{Sandy Maguire}
\date{}

\begin{document}

\frontmatter
\maketitle

\tableofcontents

\linenumbers

\show{\chapter}

\chapter*{Preface}
\addcontentsline{toc}{chapter}{Preface}

The wisdom of our elders has always been a fascinating idea to me.

In my early twenties, I used to play a game. Whenever I had some time alone with
someone older than me, I'd ask them the question: "what's something you wished
you knew when you were my age?" The goal was one day to write a book about the
answers I got, but it became increasingly evident that nobody would read such a
book. The reason was simple: the things people would answer were always one of
two things.

"I wish I had been more confident."

"I wish I had known to start investing."

Good advice to be sure, but not exactly book-worthy. Everybody already knows
they should be investing (whether or not they do is a different beast
altogether), and telling someone to be more confident isn't very actionable. If
someone isn't already confident, telling them to be more so doesn't actually
help. If anything, it might hurt.

\todo{reword this puppy} The concept of a quarter-life crisis is a relatively
new one, identified as such possibly only with Generation X. Since then, it
seems only to be growing in prevalence, particularly among young North American
professionals. According to the first Google result I came across, something
like 75\% of Millennial respondents on a Linkedin survey (aka "yuppies") said
they had experienced or were experiencing a quarter-life crisis.

Although I didn't respond to the Linkedin survey, I found myself (and the
majority of my friends) struggling through a crisis for several years. I was
jumping from job to job, from city to city, trying my best to randomly select
the proper combination of variables I thought would bring the much-desired
meaning to my life. It goes without saying that randomly flipping switches did
nothing to assuage my problems.

In desperation, I took a month off of work and went traveling. It's cliche, I
know, but there might be a good reason why people look to traveling in an
attempt to find themselves. What I found, however, was just a bunch of people as
lost as I was. And we talked, sometimes for hours, helping one another over the
obstacles we'd already worked out for ourselves.

The idea of a sudden epiphany, I think, is mostly a myth. Big ideas don't come
from nowhere, they're constructed and refined, slowly and surely. The seeds of
them come from a chance utterance of an acquaintance, or from a random passage
in a book. Big ideas are refined by going down to the cafe, spending a lot of
time writing in your notebook, and buying more coffees than is likely healthy.
In other words, you need to be open to new ideas, and when you see the
glimmering of something great, you need to be prepared to dive into the workshop
of your mind and hammer it into something worthwhile.

This was the insight I was missing. Instead of randomly changing variables in my
life hoping for something more, I should have been biding my time with a
notebook and a cup of strong coffee. I should have been approaching my life with
the same diligence I apply to my work.

\emphasis{We can be the authors of our own life stories.} It sounds obvious in
retrospect, but without a doubt, this is the most influential idea I've ever
come across. Hearing the idea isn't enough though; it's internalizing it that
really counts. That's the hard part.

Without realizing it, all those years ago when I was asking people what they
wished they had known, I was looking for a particular answer. I didn't know what
it was, but there was something missing in my soul, something I was desperately
trying to find. It's this. We can be the authors of our own life stories.

We can, but it's hard. The system isn't exactly set up to help us do that.
Western culture, for all its good points, doesn't empower its adherents in this
way. \todo{empower twice} The American Dream is enticing -- if you work hard
enough, you can accumulate great wealth -- but this presupposes that great
wealth is a worthy thing to accumulate, and leads us to believe that working
hard is virtuous.

The American Dream is ubiquitous in North American culture, to the extent that
it's often invisible. While it's not a bad thing \foreign{per se}, we should
always be hesitant about reasoning in the presence of unseen influences. As a
cultural factor, it's clearly been successful, but we should keeping in mind
that it's very clearly a product of the Industrial Revolution.

Recall that this was a time characterized by working insanely long hours under
terrible conditions in order to stamp out indistinguishable goods, all in order
to become richer than the total sum of humans previously. Basing our society,
and by extension a large part of our identities, around the idea that people are
fungible, that the economy should be grown for its own sake, and that hard work
is desirable, seems, well, unwise to put it gently.

This strikes me as the primary cause of quarter-life crises. I know too many
people who work jobs they hate, in order to make money they don't need, so they
can buy things they don't actually want. To a large degree, our society defines
success by how big our house is, what car we drive, how recently our last
promotion was, and by how successful our children are by the same metrics.


\mainmatter


\chapter{Introduction}

\begin{aquote}{Chris Kelvin, Solaris, 2002}
  I followed the current. I was silent, attentive. I made a conscious effort to
  smile, nod, stand, and perform the millions of gestures that constitute life
  on earth. I studied these gestures until they became reflexes again.
\end{aquote}

\qquad

\qquad

I'm looking out the window from my third floor location, hidden somewhere in the
depths of the university library. A few years back I accidentally stumbled in
here, and ever since it's been the place I come to sit and think.

The students down there are running around, presumably headed off to their next
class. This university is known for its philosophy program. Without anything
else to go on, I'd guess the majority of them would appreciate what I'm doing
here. Thinking.

Perhaps it's not very charitable to them, but I can't help being reminded of
ants as I watch. The students mill about, and although each of them
independently lives their own life, there is a sort of emergent organization
visible from where I sit.

There's a kind of clockwork to it. Every hour and a half, the otherwise silent
courtyard below me explodes with activity. Groups of people form for the few
minutes they can steal between classes, before disbanding. It's structured, yet
chaotic.

To a large degree, these people understand one another. They worry about the
same assignments, about getting home to see their friends and family for the
weekend, about whether or not their finances will allow them to party tonight
and still have enough to by the deluxe instant ramen next week, about what kinds
of work they'll find after university with their shiny new philosophy degrees.

Not only do they exist in the same context now, but they'll probably stay in
them. When I graduated from university, more than half of my cohort all moved to
the same city. Many of us found jobs together. We didn't feel the need to
integrate into our new home, because we had brought our old one with us. Whether
better or for worse, it's the shared contexts like these which build
communities.

And for many people, these organic communities are enough. They're familiar,
comfortable, safe. In many respects, they're boring, and I think that's a big
part of the appeal. But despite all pretenses, the boring, safe route isn't the
only way to live. It isn't necessarily the best way, either.

For as long as I can remember, a powerful driving factor in my personal
development has been the motto of \textbf{the best things in life are outside of
your comfort zone.} Statistically speaking, this is almost certainly true. The
space of human experience is so large compared with the things we have time to
do in our life, that almost by necessity, we're not going to discover the best
things that existence has to offer us. It's a litany I'm now trying to live by.

If you're anything like me, you probably have countless stories where your brain
talked you out of doing something you wanted to. Maybe it was "I can't go talk
to that pretty girl standing by herself looking bored; I don't want to
inconvenience her," or perhaps something along the lines of "I shouldn't apply
for this job, because I know I won't get it." Our brains are duplicitous things,
and without constant checks and balances, they will usually lead us astray.

It's my desire to find the best things in life that has lead me here, to this
chair in the philosophy department on the third floor of this library. It's
quiet here. It's a good place to think. The students below have left the
courtyard, back to their classes. I'll be the first person to admit that I
haven't yet found the best things in life, nor am I anywhere close, nor will I
likely ever get there. But at least I'm looking.

I think that's as good a start as any.

Too many people, I think, are held back by nothing but their own head. "I can't
do X. I'm not good/smart/talented enough. I'm not the kind of person who does
X." It might be reassuring, or possibly terrifying, to realize that most of the
people out there who try things don't feel like they're good enough either.
Usually the only difference is that people out there do things are the people
who try. Impostor syndrome at its finest.

Am I the right person to be writing a book on lifestyle design? Maybe. Maybe
not. Who can say, really? I'm just some guy who tried something different, and
who turned out to like it a hell of a lot more than the life he was being sold
by the standard institutions of Western culture. But in my opinion, it's a
worthwhile exercise, if only on the basis that it's outside of my comfort zone,
and therefore possibly one of the best things in life.

You never know until you try.

This is a book about trying.


\chapter{Starting Again}

In January of 2018, at the age of 27, I made the unconventional decision to
declare myself "retired indefinitely." This was somewhat of a shock to most
people in my life.

On paper, my life was going about as well as could be expected. I had a
prestigious degree from a prestigious university. I had worked at several of the
top companies in the world, and had strong recommendations from each of them. I
was well on my way to becoming a distinguished member of my field. By all
accounts, I had a bright career ahead of me.

But there was just one problem; that wasn't the life I wanted. What I wanted, I
wasn't sure, but it certainly wasn't that. The evidence of it was all around me;
with coworkers twenty years my senior working the same jobs as me.

I could all too easily imagine myself waking up at 50, wondering where the hell
my life had gone. I'd consult my bucket-list, and realize that I hadn't made any
progress on it in years, and now I had too many responsibilities and that my
health was too too poor to do the things I'd always promised myself I'd get
around to.

No thanks.

Despite my best intentions, I found myself having been sucked into the system. I
was working for the man in a job I hated. I was doing it to make money that I
didn't need, in order to buy things I didn't actually want. Like many others, I
had convinced myself that "this is just what being an adult is." I had resigned
myself to this unsatisfying life that I had unintentionally crafted for myself.

It wasn't the life I'd have designed for myself if I had been trying. If I was
being honest with myself, , I wasn't entirely sure what that life would look
like, but I was willing to find out. One of the biggest problems of modern
society is that it doesn't give us many chances to try different things. Sure,
you can order the salmon instead of the chicken when you go out to dinner, but
it's not like you can just decide to see what happens if we give up democracy
for a weekend or something.  The existing order is too deep-rooted and powerful
for that.

But I wasn't happy where I was, and I didn't know where I wanted to be. The only
solution in that case is to experiment, and so I considered my retirement to be
an exercise in the scientific method.

The way I looked at it, it wasn't a particularly costly test. Worst case, it
wouldn't work out, and I could return to the society I left behind with my tail
between my legs, begging to be let back in. I had constructed that life once, by
accident nevertheless, so presumably I could rebuild it if I were actually
\emphasis{trying}.

Like many people, I have a bad habit of not being particularly good at putting
my money where my mouth is. \todo{terrible sentence} I'm sure that you hear
complaints weekly from your friends about their bad jobs, relationships or
what-have-you, but despite their comments that they're going to make a change,
things rarely do. Out of fear of continuing to live by status-quo forever, I
decided to pull the trigger, and quit my job immediately. I didn't have a plan;
I just knew I needed to make a change.

Three months later, I found myself living in Eastern Europe, spending 100\% of
my time working on meaningful personal projects, hanging out with people I love,
and accumulating a wealth of experiences I couldn't have even imagined existing
before moving here. It's been the best thing I've ever done with my life.

Maybe none of my story has resonated with you. If not, that's OK. This book
probably isn't for you, and I won't be offended if you put it down right here
and now. But maybe something has hit a chord in your spirit. Maybe you too have
some unnamed dissatisfaction in the direction your life has been taking.

Fortune favors the bold. If you're unhappy with some facet of your existence,
\emphasis{do something about it.} Don't let it fester within you, exploding in
twenty years in the form of a mid-life crisis. At that point it might be too
late. So take a chance, get outside of your comfort zone, and let's go for one
hell of a trip.


\chapter{Taking Stock}

A few years back, I attended some kind of hokey meditation retreat. I didn't
remember signing up, but it was work-sanctioned so I decided to take it; it
seemed like a better way to spend a few days than slinging computer programs
around.

Throughout the retreat, I remember there being a lot of focus on breathing and
paying attention to what we were eating, and silly things like that. To be
honest, not much of it stuck with me. There was one exception.

The exercise that I remember was a meditation to visualize one of our perfect
futures. The instructor was clear to emphasize that we were looking for
\emphasis{one} of our perfect futures, implying there might be many, that it
might not be unique. This was already somewhat of a head-trip to me; I'd never
realized that there might be many, exceptionally different futures for me, all
of which I might describe as being perfect.

This was an epiphany, and being phrased like this allowed me to be open to
futures I would have otherwise ignored as being "out of hand." The human brain
is one that likes to feel smart, powerful and in-control of its own destiny, and
so if I had to choose \emphasis{the} perfect future, it would have been in-line
with all of the life choices I had made up to that point.

But instead I was given the great gift of needing to look only for any of my
perfect futures. Without much forethought, I put my pen to the page and just
started writing the stream of consciousness that had laid dormant but was now
flowing rapidly. I wrote without stopping for about five minutes, and fell
silent. The contents of my mind that now lay upon that paper honestly surprised
me.

The future I had come up with seemed very alien. I had envisioned myself living
a simple life in a commune on top of a mountain somewhere. I spent my time
growing food, talking with my neighbors, and tinkering with trains when I could.
Strange thoughts to be certain, but upon rereading them I was sure that this was
in fact a future with which I would be satisfied.

Curious, isn't it? That I could so easily come up with such a drastically
different way of living, one in which I was certainly happier than I was now?
It's almost as if I knew all along what I wanted, but as a defense mechanism,
had kept below the level of consciousness.

And so, if you're really and truly serious about wanting to change your life,
I'm going to make the very strong suggestion that you spend ten minutes thinking
about one of your ideal futures, and writing it down. If you are anything like
me when I read books, you might be thinking to yourself that you can just skip
the exercise and continue reading on. Don't do this. The lesson from education
is very clear about this: it's the homework that causes learning, not the
lecture itself. If you skip this (or any exercise recommended in this book) you
are doing yourself a disservice. Do the exercise.

The exercise is this: take ten minutes to write down two or three of your ideal
futures. Don't think about it, just set a timer, start writing, and stop when
the bell rings. If you find yourself describing a life that sounds a lot like
the one you're currently living, add in the first twist that comes to mind and
see where it takes you. The twist might be "what if I lived in a different
country" or "what if I had followed that passion I had as a child" or what have
you. If you are describing your current life, you are not being honest with
yourself, and following through with it will only delude you into fortifying the
status-quo with which you are unhappy.

Ready? Set your alarm for ten minutes. Start now.

Done.

Did you come up with anything surprising? It might not have been
earth-shattering, but I'll bet that upon review, at least one thing on your list
wasn't something you'd have expected to be there.

If so, that's great. I want you to focus on that surprising thing. The point of
the exercise is to realize that even if you're happy with the way things are
going, it's not the only way that things \emphasis{could} go. Keep this in mind.


\chapter{Minimalism}

keep your identity small
* or if not possible, at least add to your identity "i take chances"
* and "i am capable of changing"

keep your possessions small
* this is hard for americans
* the things you own are not what define you
  * well they sort of do, but only in a BAD way
  * they keep you rooted
* my story about shluffing boxes around forever

maybe this should be called "mindset"


\chapter{Finding a Space}

\chapter{Financing Yourself}

holy shit america is so fucking expensive
i live on about \$35 dollars a day, and i am living very lavishly


\chapter{Putting Down Roots}

rich american + cultural appropriation? no, go fuck yourself

if you make an effort to fit in, people will love you.

learn the local language, follow the local culture, make friends with locals

the goal is to integrate. it's ok to have expat friends, but keep them under 50\%


\end{document}

