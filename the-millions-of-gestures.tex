\documentclass[]{book}
\usepackage{todonotes}
\usepackage [english]{babel}
\usepackage [autostyle, english = american]{csquotes}
\MakeOuterQuote{"}


\def\signed #1{{\leavevmode\unskip\nobreak\hfil\penalty50\hskip2em
  \hbox{}\nobreak\hfil(#1)%
  \parfillskip=0pt \finalhyphendemerits=0 \endgraf}}

\newsavebox\mybox
\newenvironment{aquote}[1]
  {\savebox\mybox{#1}\begin{quote}}
  {\signed{\usebox\mybox}\end{quote}}

%opening
\title{The Millions of Gestures}
\author{Sandy Maguire}

\begin{document}

\frontmatter
\maketitle

\tableofcontents

\mainmatter

\chapter{Introduction}

\begin{aquote}{Chris Kelvin, Solaris, 2002}
  I followed the current. I was silent, attentive. I made a conscious effort to
  smile, nod, stand, and perform the million of gestures that constitute life on
  earth. I studied these gestures until they became reflexes again.
\end{aquote}


\chapter{Starting Again}

In January of 2018, at the age of 27, I made the unconventional decision to
declare myself "retired indefinitely." This was somewhat of a shock to most
people in my life.

On paper, my life was going about as well as could be expected. I had a
prestigious degree from a prestigious university. I had worked at several of the
top companies in the world, and had strong recommendations from each of them. I
was well on my way to becoming a distinguished member of my field. By all
accounts, I had a bright career ahead of me.

But there was just one problem; that wasn't the life I wanted. What I wanted, I
wasn't sure, but it certainly wasn't that. The evidence of it was all around me;
with coworkers twenty years my senior working the same jobs as me.

I could too easily imagine myself waking up at 50, wondering where the hell my
life had gone. I'd consult my bucket-list, and realize that I hadn't made any
progress on it in years, and now I had too many responsibilities and too poor
health to do the things I'd always promised myself I'd do.

No thanks.

Despite my best intentions, I found myself having been sucked into the system. I
was working for the man in a job I hated. I was doing it to make money that I
didn't need, in order to buy things I didn't actually want. Like many others, I
had convinced myself that "this is just what being an adult is." I had resigned
myself to this unsatisfying life that I had unintentionally crafted for myself.

It wasn't the life I'd have designed for myself if I had been trying. To be
honest, I wasn't entirely sure what that life would look like, but I was willing
to find out. One of the biggest problems of modern society is that it doesn't
give us many chances to try different things. Sure, you can order the salmon
instead of the chicken when you go out to dinner, but it's not like you can just
decide to see what happens if we give up democracy for a weekend or something.
The existing order is too deep-rooted and powerful for that.

But I wasn't happy where I was, and I didn't know where I wanted to be. The only
solution in that case is to experiment, and so I considered my retirement to be
an exercise in the scientific method.

The way I looked at it, it wasn't a particularly costly test. Worst case, it
wouldn't work out, and I could return to the society I left behind with my tail
between my legs, begging to be let back in. I had constructed that life once, by
accident nevertheless, so presumably I could rebuild it if I were actually
\textit{trying}.

Like many people, I have a bad habit of not being particularly good at putting
my money where my mouth is. \todo{terrible sentence} I'm sure that you hear
complaints weekly from your friends about their bad jobs, relationships or
what-have-you, but despite their comments that they're going to make a change,
things rarely do. Out of fear of continuing to live by status-quo forever, I
decided to pull the trigger, and quit my job immediately. I didn't have a plan;
I just knew I needed to make a change.

\end{document}
